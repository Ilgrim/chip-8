% Copyright 2015-2016 Matthew Mikolay.
%
% This program is free software: you can redistribute it and/or modify
% it under the terms of the GNU General Public License as published by
% the Free Software Foundation, either version 3 of the License, or
% (at your option) any later version.
%
% This program is distributed in the hope that it will be useful,
% but WITHOUT ANY WARRANTY; without even the implied warranty of
% MERCHANTABILITY or FITNESS FOR A PARTICULAR PURPOSE.  See the
% GNU General Public License for more details.
%
% You should have received a copy of the GNU General Public License
% along with this program.  If not, see <http://www.gnu.org/licenses/>.

\documentclass{article}
\usepackage[utf8]{inputenc}
\usepackage[letterpaper, portrait, margin=1in]{geometry}
\usepackage{vhistory}
\usepackage{multicol}
\usepackage[backend=biber,
style=numeric,
sorting=nty
]{biblatex}
\usepackage{titlesec}
\usepackage{microtype}

\title{CHIP-8 Extensions Reference}
\author{Matt Mikolay}
\date{December 21, 2015}

% These lines should be uncommented when using versionhistory
% \title{CHIP-8 Extensions Reference\\{\Large Revision \vhCurrentVersion}}
% \date{\vhCurrentDate}

% For Baskerville font
\usepackage{librebaskerville}
\usepackage[T1]{fontenc}

% Don't indent paragraphs
\setlength{\parindent}{0em}

% Define a command for formatting subtitles
\newcommand{\subtitle}[1] {{\hfill \small{\emph{#1}}}}

% Attach a bibliography to this document
\addbibresource{references.bib}

% Adjust spacing before and after section titles
\titlespacing\section{0pt}{14pt plus 4pt minus 2pt}{1pt plus 1pt minus 1pt}

\begin{document}

\maketitle

\begin{multicols}{2}
Since its introduction in the 1970s, the CHIP-8 programming language has spawned a variety of dialects and descendant languages. This document aims to briefly describe the various programming languages related to CHIP-8. Entries are listed chronologically. \\

This document should not be considered comprehensive; certain CHIP-8 descendants may be missing. To suggest a revision, please contact the author via mattmik.com.

\section*{CHIP-8C \subtitle{RCA, 1978}}
Described as ``the color-language addition to CHIP-8,'' CHIP-8C promised control of three background colors and eight foreground colors in conjunction with RCA's VIP Color Board. It is not known whether this language was released publicly, or if it eventually grew into RCA's CHIP-8X. \cite{chip8c}

\section*{CHIP-8I \subtitle{Rick Simpson, 1978}}
A modification to the original CHIP-8 interpreter that provides three new instructions to support hardware I/O. \cite{chip8i}

\section*{CHIP-10 \subtitle{Ben H. Hutchinson, Jr., 1979}}
A modified version of CHIP-8 providing an expanded screen resolution of 128 x 64. \cite{chip10}

\section*{CHIP-8 II \subtitle{Tom Swan, 1979}}
A modification to the original CHIP-8 interpreter, and an optional extension to Rick Simpson's CHIP-8I language, that adds another instruction to read from the VIP's input port. This change was made to allow for the creation of two-player games controlled by an ASCII keyboard. \cite{chip8ii}

\section*{HI-RES CHIP-8 \subtitle{Tom Swan, 1980}}
A modified version of CHIP-8 providing an expanded screen resolution of 128 x 64 and increased speed. \cite{hires}

\section*{CHIP-8III \subtitle{John Chmielewski, 1980}}
A modified version of CHIP-8 aimed at providing the functionality of both Rick Simpson's CHIP-8I and Tom Swan's CHIP-8 II while maintaining compatibility with the original CHIP-8 language. \cite{chip8iii}

\section*{CHIP-8E \subtitle{Gilles Detillieux, 1980}}
A rewrite of the original CHIP-8 interpreter that adds fourteen additional instructions and support for hardware I/O. \cite{chip8e}

\section*{CHIP-8X \subtitle{RCA, 1980}}
An ``expanded version of the original CHIP-8 interpreter'' meant for use with a series of expansion modules marketed by RCA for the COSMAC VIP system: the VP-590 Color board, the VP-595 Simple Sound board, and the VP-580 Expansion Keypad. The CHIP-8X language adds color graphics, extended sound capabilities, and support for a second keypad to the original CHIP-8 language. \cite{gamemanual, chip8x}

\section*{CHIP-8Y \subtitle{Bob Casey, 1981}}
A modified version of CHIP-8 supporting hardware I/O while maintaining compatibility with the original CHIP-8 language. \cite{chip8y}

\section*{CHIP-8M \subtitle{Steven V. Gunhouse, 1982--1983}}
A modified version of CHIP-8 providing instructions to output International Morse Code tones. \cite{chip8m}

\section*{SUPER-CHIP \subtitle{Erik Bryntse, 1991}}
An expansion of the CHIP-8 instruction set first introduced as a modification of Andreas Gustafsson's CHIP-8 emulator for the Hewlett-Packard HP 48 series of calculators. This CHIP-8 variant offers a higher screen resolution, larger sprites, larger fonts, screen scrolling, and more. It is also known as ``S-CHIP''. \cite{schip}

\section*{MEGA-CHIP8 \subtitle{Martijn Wenting, 2007}}
A modern extension to the CHIP-8 language providing an expanded screen resolution, larger sprites, color graphics, and digital sound. \cite{megachip2, megachip1}

\section*{XO-CHIP \subtitle{John Earnest, 2014}}
A modern extension to the CHIP-8 language introduced alongside the Octo IDE. This variant supports additional save and load instructions, color graphics, programmable audio, and screen scrolling. \cite{xochip}

\printbibliography

\end{multicols}

% Uncomment when using versionhistory
% \begin{versionhistory}
%     \vhEntry{1.0}{Dec. 15, 2015}{Matt Mikolay}{Created document.}
% \end{versionhistory}

\end{document}
